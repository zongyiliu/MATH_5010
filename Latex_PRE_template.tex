\documentclass[letterpaper]{article} 
\usepackage[utf8]{inputenc}
\linespread{0.85}
\usepackage[T1]{fontenc}
\usepackage{amsmath}
\usepackage{amsfonts}
\usepackage{amssymb}
\usepackage{array}
\usepackage{booktabs}
\usepackage{hyperref}
\usepackage[version=4]{mhchem}
\usepackage{stmaryrd}
\usepackage[dvipsnames]{xcolor}
\colorlet{LightRubineRed}{RubineRed!70}
\colorlet{Mycolor1}{green!10!orange}
\definecolor{Mycolor2}{HTML}{00F9DE}
\usepackage{graphicx}
\usepackage{amsmath}
\usepackage{graphicx}
\usepackage{capt-of}
\usepackage{lipsum}
\usepackage{fancyvrb}
\usepackage{tabularx}
\usepackage{listings}
\usepackage[export]{adjustbox}
\graphicspath{ {./images/} }
\usepackage[utf8]{inputenc}
\usepackage[english]{babel}
\usepackage{float}
\usepackage{lipsum}
\usepackage{graphicx}
\usepackage{float}
\usepackage[margin=0.7in]{geometry}
\usepackage{amsmath}
\usepackage{graphicx}
\usepackage{capt-of}
\usepackage{tcolorbox}
\usepackage{lipsum}
\usepackage{graphicx}
\usepackage{float}
\usepackage{listings}
\usepackage{hyperref} 
\usepackage{xcolor} % For custom colors
\lstset{
	language=Python,                % Choose the language (e.g., Python, C, R)
	basicstyle=\ttfamily\small, % Font size and type
	keywordstyle=\color{blue},  % Keywords color
	commentstyle=\color{gray},  % Comments color
	stringstyle=\color{red},    % String color
	numbers=left,               % Line numbers
	numberstyle=\tiny\color{gray}, % Line number style
	stepnumber=1,               % Numbering step
	breaklines=true,            % Auto line break
	backgroundcolor=\color{black!5}, % Light gray background
	frame=single,               % Frame around the code
}
\usepackage{float}
\usepackage[]{amsthm} %lets us use \begin{proof}
	\usepackage[]{amssymb} %gives us the character \varnothing
	
	\title{Homework 1, MATH 4155}
	\author{Zongyi Liu}
	\date{Wed, Jan 28, 2026}
	\begin{document}
	\maketitle
\section{Question 1}

\textbf{Answer}

\clearpage

\section{Question 2}

\textbf{Answer}

\clearpage

\section{Question 3}

\textbf{Answer}

\clearpage

\section{Question 4}

\textbf{Answer}

\clearpage

\section{Question 5}

\textbf{Answer}

\clearpage

\section{Question 6}

\textbf{Answer}

\clearpage

\section{Question 7}

\textbf{Answer}

\clearpage

\section{Question 8}

\textbf{Answer}

\clearpage

\section{Question 9}

\textbf{Answer}

\clearpage

\section{Question 10}

\textbf{Answer}

\clearpage

\section{Question 11}

\textbf{Answer}

\clearpage

\section{Question 12}

\textbf{Answer}

\clearpage

\section{Question 13}

\textbf{Answer}

\clearpage

\section{Question 14}

\textbf{Answer}

\clearpage

\section{Question 15}

\textbf{Answer}

\clearpage

\section{Question 16}

\textbf{Answer}

\clearpage

\section{Question 17}

\textbf{Answer}

\clearpage

\section{Question 18}

\textbf{Answer}

\clearpage

\section{Question 19}

\textbf{Answer}

\clearpage

\section{Question 20}

\textbf{Answer}

\clearpage
		
	\end{document}
